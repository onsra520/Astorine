\documentclass{article}

\usepackage{lmodern}
\usepackage{graphicx}
\usepackage{ragged2e}

\title{Personal computer recommendations: Using Neural Networks and Object Detection \\ {\large FPT University, Ho Chi Minh Campus, Vietnam}}
\author{Cao Trần Tiến}
\date{}

\begin{document}

\maketitle

\begin{flushleft}
    \textbf{Abstract:} đây là Abstract
\end{flushleft}

\begin{flushleft}
    \textbf{Keywords:} Deep Learning, Machine Learning, Computer Vision, Vision Transformer, Self-Supervised Learning, Agricultural Technology.
\end{flushleft}

\section{Introduction}
Đây là tài liệu đầu tiên của tôi viết bằng

\section{Related work}

\section{Methodology}

\subsection{Dataset}
\begin{justify}
Two datasets, Personal Computer Information and Personal Computer Image, were utilized in this study.
\end{justify}

\begin{justify}
\textbf{Personal Computer Information Dataset:}  
This dataset contains numerical specifications of personal computers, including details such as GPU, CPU, RAM, storage, and other technical attributes.
\end{justify}

\begin{justify}
\textbf{Personal Computer Image Dataset:}  
\begin{itemize}
    \begin{justify}
    \item \textbf{Number of Images:} 
    3,108 images.
    \end{justify}
    \begin{justify}
    \item \textbf{Categories:} 
    Up to 15 classes of laptop models.
    \end{justify}
    \begin{justify}
    \item \textbf{Image Characteristics:} 
    Images include multiple laptop sides with a filtered background, ensuring consistency for analysis.
    \end{justify}
    \begin{justify}
    \item \textbf{Image Size:} 
    All images are resized to 600×600 pixels to meet the model’s input requirements and reduce computational load.
    \end{justify}

\end{itemize}
\end{justify}

\section{Evaluation}

\section{Comparative Analysis}

\section{Result}

\section{Discussion}

\end{document}
